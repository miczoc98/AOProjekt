\subsection{Segmentacja w projekcie}
    Isotnym elementem naszej aplikacji jest segmentacja. Musieliśmy napisać skrypt w taki sposób, by uzyskać odczytany tekst z obrazka z podziałem na linie, słowa i litery. Docelowo skrypt zwraca tablice, która zawiera pojedyńcze litery, spacje lub znak nowej linii.
    Podział na linie zrealizowaliśmy sumując wartości danego wiersza macierzy obrazu po binaryzacji. Jeśli wartość wiersza była większa od zera to oznaczało, że jest to część linii.
    Następnie każdą linie tekstu analizowaliśmy osobno. Podział na słowa, zrealizowaliśmy przy użyciu dylatacjii tak by znaki się połączyły i wyznaczyły słowo. Wartość dylatacjii określiliśmy doświadczalnie dla testowanych przykładów.
    W trakcie realizacjii segmentacji pojawiły się problemy, np. kropka nad i (jak ją zachować, rozwiązane przez BoundingBox), czy też problem kerningu.
