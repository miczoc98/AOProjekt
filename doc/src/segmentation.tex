\subsection{Segmentacja w projekcie}
    Isotnym elementem aplikacji jest segmentacja. Napisany skrypt uzyskuje odczytany tekst z obrazka z podziałem na linie, słowa i litery. Docelowo skrypt zwraca tablice, która zawiera pojedyncze litery, spacje lub znak nowej linii.
    Podział na linie zrealizowany jest przez sumowanie wartości danego wiersza macierzy obrazu po binaryzacji. Jeśli wartość wiersza jest większa od zera to znaczy, że jest to część linii.
    Następnie każda linia tekstu podlega analizie osobno. Podział na słowa, realizuje użycie dylatacjii tak by znaki się połączyły i wyznaczyły słowo. Wartość dylatacjii została określona doświadczalnie dla testowanych przykładów.
    W trakcie realizacjii segmentacji pojawiły się problemy, np. kropka nad i (jak ją zachować), czy też problem kerningu.