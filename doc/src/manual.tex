\section{Obsługa programu}
	W celu uzyskania wyniku w programie należy wykonać następujące kroki:
	\begin{enumerate}
		\item Wybór obrazu dla którego nastąpi rozpoznanie tekstu prze pomocy przycisku \textbf{Select Image}. Obraz powinien zostać wyświetlony w lewej górnej części aplikacji.
		\item Wybór czcionki odpowiadającej tej widocznej na obrazie lub, w przypadku nieznanej czcionki wartości \textbf{Other}.
		\item Dobór odpowiednich parametrów dla binaryzacji obrazu. 
			Dla trybu \textbf{manual} wiąże się to z zaznaczeniem odpowiednich warstw koloru (R, G, B) i wyborem wartości granicznej przy pomocy suwaka \textbf{Threshold}.
			Zbinaryzowany obraz jest wyświetlony poniżej oryginalnego.			
			Wynikowy obraz powinien zawierać białe litery na czarnym tle - jeżeli jest odwrotnie należy zaznaczyć opcję \textbf{Color Inversion}.
		\item Przeprowadzenie operacji rozpoznania tekstu przy użyciu przycisku \textbf{Run OCR}. Wynikowy tekst zostanie wypisany w prawym panelu aplikacji.
	\end{enumerate}