\subsection{Rozpoznawanie tekstu}

W celu zrealizowania zagadnienia rozpoznawania tekstu, utworzono najpierw odpowiednie zbiory znaków dla każdej z obsługiwanych czcionek (zaimplementowano obsługę czterech czcionek: Arial, Helvetica, Times New Roman, Purisa). Zbiór znaków danej czcionki reprezentują trzy pliki .png, zawierające odpowiednio: duże litery, małe litery oraz pozostałe znaki. Obrazy te są wczytywane do programu, a następnie pozyskiwane są z nich poszczególne znaki (użyto w tym celu przejście do odcieni szarości, binaryzację oraz segmentację). W kolejnym kroku łączone są segmenty znaków "nieciągłych" (np. 'i', 'j', ':'), a ostatecznie obraz każdego ze znaków zostaje odpowiednio przeskalowany do kwadratu o razmiarze 50x50.\\

Aby umożliwić jak najdokładniejsze rozpoznawanie tekstu, który nie został napisany żadną z czterech obsługiwanych czcionek, przygotowano również "uśredniony" zbiór znaków. Został on utworzony poprzez obliczenie średniej arytmetycznej dla każdego piksela każdego obrazu znaku z czcionek Arial, Helvetica i Times New Roman. Następnie, z racji na binarny charakter przygotowanych zbiorów znaków, wartość danego piksela została odpowiednio zaokrąglona.\\

Do rozpoznawania poszczególnych znaków wykorzystano korelację. Podany na wejście funkcji obraz znaku, który ma zostać rozpoznany (uprzednio przeskalowany do kwadratu 50x50) jest porównywany kolejno z każdym obrazem z przygotowanego zbioru pod kątem korelacji. Następnie szukane jest maksimum z wyznaczonych wartości korelacji. Znak wejściowy zostaje zakwalifikowany jako ten, dla którego wartość korelacji była największa. Ostatecznie zwracany jest rozpoznany znak w formie tekstowej.
