\section{Testowanie poprawności działania programu}
Do programu dołączono kilka obrazów testowych (po dwa dla każdej z obsługiwanych czcionek oraz jeden z tekstem napisanym nieobsługiwaną czcionką), zarówno czarno-białych jak i kolorowych,  pozwalających na przetestowanie aplikacji. Po wykonaniu serii testów można stwierdzić, że zaimplementowany mechanizm binaryzacji dobrze radzi sobie z obrazami kolorowymi, a segmentacja dzieli tekst w zamierzony sposób. Jeśli zaś chodzi o samo rozpoznawanie wejściowych znaków, również i w tym aspekcie wynik można uznać za zadowalający, jednak pojawiają się pewne niedoskonałości. Na przykład nie udało się rozpoznać znaku apostrofa, który jest przez aplikację traktowany jako przecinek. Dzieje się tak dlatego, że w programie nie uwzględniono położenia znaku w linii, a jedynie jego kształt (a pod tym względem apostrof i przecinek są niemal jednakowe). Kolejnym błędem, który udało się zaobserwować jest traktowanie znaku kropki w czcionce Arial jako znaku "A". Może się również zdarzyć, że program pomyli dużą literę "I" z małą "l", ze względu na ich podobieństwo. W przypadku testowego obrazu przedstawiającego tekst napisany nieobsługiwaną czcionką również pojawia się kilka błędnych interpretacji, lecz wynikowy tekst jest wciąż w miarę czytelny. Biorąc jednak pod uwagę, że do rozpoznania tego tekstu użyto połączenia trzech innych czcionek, można uznać ten rezultat za satysfakcjonujący.\\

W aplikacji nie udało się zaimplementować obsługi tekstu napisanego kursywą, jak również tekstu skierowanego pod kątem różnym od orientacji poziomej. Program może sobie również nie poradzić, jeśli w tekście występuje kerning (obrazy testowe zostały przygotowane w ten sposób, żeby kerning się nie pojawił). Są to aspekty, nad którymi możnaby popracować przy dalszym rozwoju aplikacji.


